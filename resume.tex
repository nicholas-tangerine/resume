\documentclass[a4paper,10pt]{article}

% Packages
\usepackage[margin=0.5in]{geometry}
\usepackage{enumitem}
\usepackage[hidelinks,colorlinks=true,urlcolor=blue]{hyperref}
\usepackage{titlesec}

% Formatting tweaks
\setlist[itemize]{noitemsep, topsep=0pt, leftmargin=1.5em}
\setlength{\parindent}{0pt}
\titleformat{\section}{\large\bfseries}{}{0pt}{}[\titlerule]
\titleformat{\section}{\large\bfseries}{}{0pt}{}[\titlerule]
\titlespacing*{\section}{0pt}{0.5em}{0.3em}

\begin{document}
\pagenumbering{gobble}

%-------------------- HEADER --------------------
\begin{center}
    {\LARGE \textbf{Nicholas Tang}} \\ [0.5em]
    tangnicholas26@gmail.com \,|\, +1 (925) 660-5921 \\ [0.3em]
    \href{https://nicholas-tangerine.github.io}{Portfolio} \,|\, 
    \href{https://github.com/nicholas-tangerine}{GitHub} \,|\, 
    \href{https://linkedin.com/in/nicholas-tangerine}{LinkedIn}
\end{center}

%-------------------- SUMMARY --------------------
\section*{Summary}
Computer Science student (GPA 3.9) with hands-on experience in embedded
systems, physics-based simulation, and teaching. Skilled in C/C++, Python, and
system-level programming. Fascinated by the intersection of computing,
aerospace, and physics, with experience spanning NASA sustainability
infrastructure, CAN bus systems, and cloud/wind pattern tracking on Jupiter.

%-------------------- EDUCATION --------------------
\section*{Education}
\textbf{UC Santa Cruz} \hfill Sept 2024 -- Jun 2027 \\
B.S. Computer Science with Electrical Engineering Minor \hfill GPA: 3.9 \\
\textit{Relevant Coursework:} Data Structures \& Algorithms, Intro to Electronic Circuits \\
(+ Lab), Differential Equations

%-------------------- EXPERIENCE --------------------
\section*{Experience}
\textbf{Firmware Engineer --- Formula Slug, FSAE} (C++) \hfill Sept 2024 -- Present
\begin{itemize}
    \item Implemented automatic lap counting using GPS data and converting to local coordinate \\
        plane with a flat-Earth approximation and tolerance of $\pm$ 10 meters
    \item Designed and implemented software handling communication at 500 kHz over CAN bus.
    \item Built on top of MbedOS, RTOS for STM32 MCUs; CMake with ninja for building and \\
        flashing to MCU
    \item Developed tooling to test hardware/software; collaborated with team of $\sim$100.
\end{itemize}

\textbf{Planetary Cloud Tracking Research} (C) \hfill Jun 2025 -- Present
\begin{itemize}
    \item Exploring computer vision algorithms to track wind patterns on Jupiter and other \\
        planetary atmospheres.
    \item Implementing image processing algorithms in C
    \item Soon to be parallelized using either CUDA or compute shaders
\end{itemize}

\textbf{Instruction Officer --- Google Developer Groups on Campus} \hfill Mar 2025 -- Present
\begin{itemize}
    \item Increased attendance by 30\% in Data Structures \& Algorithms sessions.
    \item Planned and hosted workshops; implemented participation-boosting initiatives.
\end{itemize}

\textbf{Vice President \& Coding Branch Leader --- BOBTutor} \hfill 2021 -- 2022
\begin{itemize}
    \item Organized and led 3 coding programs; mentored 5 new team members.
    \item Coordinated schedules and developed resources to streamline teaching.
\end{itemize}

%-------------------- PROJECTS --------------------
\section*{Projects \& Other Experience}
\textbf{Lead Systems Engineer --- NASA's NPWEE Program} \hfill June 2025 -- Aug 2025
\begin{itemize}
    \item Worked on mid-air battery swap infrastructure for electric planes.
    \item Spearheaded design and writing for the final proposal of 7 pages.
    \item Researched airspace management and aviation systems; workforce development program.
    \item Collaborated on a team of 12.
\end{itemize}

\textbf{WindowWise --- ACMHacks} (Node.js, Python) \hfill Oct 2024
\begin{itemize}
    \item Enables passive cooling systems instead of HVAC by solving heat equation
    \item Optimizes climate control while reducing energy waste by 100\%.
\end{itemize}

\textbf{Three-Body Problem Simulator} (Python) \hfill 2025
\begin{itemize}
    \item Built a numerical physics engine simulating gravitational \\
        interactions of three bodies in 2D.
    \item Implemented ODE solvers and visualization of orbital trajectories.
\end{itemize}

\textbf{Ray Tracing in One Weekend} (C++) \hfill 2025
\begin{itemize}
    \item Implemented a physically-based rendering engine in C++.
\end{itemize}

%-------------------- LEADERSHIP & ACTIVITIES --------------------
\section*{Leadership \& Activities}
\textbf{Formula Slug} --- Software + Firmware Engineering Member \\
\textbf{Association for Computing Machinery (ACM)} --- Member \\
\textbf{Google Developer Groups on Campus} --- Instruction Officer, Workshop
Organizer

%-------------------- SKILLS --------------------
\section*{Skills}
\textbf{Programming Languages:} Python, C/C++, Java, C\#, Node.js \\
\textbf{Programming Tools:} Linux, Git/GitHub, GNU Make, Vim, Valgrind, Docker \\
\textbf{Lab Tools:} Oscilloscope, Power Supply, Multimeter, Circuit Design (KiCad), Soldering
\textbf{Languages:} Python, C/C++, Java, C\#, Node.js \\
\textbf{Tools:} Linux, Git/GitHub, GNU Make, vim, Valgrind, Docker \\
\textbf{Languages:} English (Fluent), French (Conversational), Cantonese (Spoken)

\end{document}
