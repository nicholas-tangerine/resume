\documentclass[letterpaper,10pt]{article}

% Packages
\usepackage[margin=0.5in]{geometry}
\usepackage{enumitem}
\usepackage{mathtools}
\usepackage[hidelinks,colorlinks=true,urlcolor=blue]{hyperref}
\usepackage{titlesec}

% Formatting tweaks
\setlist[itemize]{noitemsep, topsep=0pt, leftmargin=1.5em}
\setlength{\parindent}{0pt}
\titleformat{\section}{\large\bfseries}{}{0pt}{}[\titlerule]
\titleformat{\section}{\large\bfseries}{}{0pt}{}[\titlerule]
\titlespacing*{\section}{0pt}{0.5em}{0.3em}

\begin{document}
\pagenumbering{gobble}

%-------------------- HEADER --------------------
\begin{center}
    {\LARGE \textbf{Nicholas Tang}} \\ [0.5em]
    tangnicholas26@gmail.com \,|\, +1 (925) 660-5921 \,|\, SF Bay Area, CA \\ [0.3em]
    \href{https://nicholas-tangerine.github.io}{Portfolio} \,|\, 
    \href{https://github.com/nicholas-tangerine}{GitHub} \,|\, 
    \href{https://linkedin.com/in/nicholas-tangerine}{LinkedIn}
\end{center}

%-------------------- SUMMARY --------------------
\section*{Summary}
Embedded firmware engineer with hands-on experience developing safety-critical
systems in C/C++ for electric vehicles. Built STM32 firmware for PID-based
traction control and regenerative braking, with robust communication over CAN
and UART. Strong focus on low-level systems, real-time constraints, and
hardware–software integration.

%-------------------- EDUCATION --------------------
\section*{Education}
\textbf{UC Santa Cruz}              \hfill Sept 2024 -- Jun 2027    \\
B.S. Computer Science, EE Minor     \hfill GPA: 3.95

%-------------------- EXPERIENCE --------------------
\section*{Experience}
\textbf{Formula Slug (Formula Society of Automotive Engineers) @ UC Santa Cruz} \hfill Sept 2024 -- Present \\
\textit{Firmware Engineer} (C++)
\begin{itemize}
    \item Developing firmware for a custom electric vehicle for the FSAE international collegiate competition.
    \item Implemented automatic lap counting using GPS data from CAN bus with accuracy of $\pm$10 meters.
    \item Managed 10 MHz UART communication with VN-200 for performant IMU and GPS data acquisition.
    \item Developed firmware for regenerative braking and PID-based traction control on STM32 MCUs.
    \item Utilized MbedOS RTOS, CMake, and Ninja for building and flashing to ARM-Cortex M cores.
    \item Engineered safety-critical power delivery systems, implementing LV undervolting detection and relay actuation.
\end{itemize}

\textbf{UCSC Earth and Planetary Sciences} \hfill Jun 2025 -- Present \\
\textit{Planetary Cloud Tracking Research} (C/C++)
\begin{itemize}
    \item Exploring computer vision algorithms to track wind patterns on Jupiter and other planetary atmospheres.
    \item Exposure to and translation between HDF5, NetCDF, TIFF, etc.
    \item Using CMake to manage complex, large-scale projects
    \item Implementing image/signal processing algorithms in C
    \item Solving Euler-Lagrange equations with numerical integration
\end{itemize}

%-------------------- PROJECTS --------------------
\section*{Projects \& Other Experience}
\textbf{NASA's Professional Development Program (NPWEE)} \hfill June 2025 -- Aug 2025 \\
\textit{Lead Systems Engineer}
\begin{itemize}
    \item Worked on mid-air battery swap infrastructure for electric planes.
    \item Spearheaded design and writing for the final proposal of 7 pages.
    \item Researched airspace management and aviation systems; workforce development program.
    \item Collaborated on a team of 12.
\end{itemize}

\textbf{Musical Auto-Transcribe DSP} \hfill Dec 2025 -- Present
\begin{itemize}
    \item Turns audio files into human readable musical notation
    \item Applying FFT algorithms, Hann windowing, and general signal denoising
\end{itemize}

\textbf{WindowWise --- ACMHacks} (Node.js, Python) \hfill Oct 2024
\begin{itemize}
    \item Enables passive cooling systems instead of HVAC by solving heat equation
    \item Optimizes climate control while reducing energy waste by 100\%.
\end{itemize}

\textbf{Three-Body Problem Simulator} (Python) \hfill 2025
\begin{itemize}
    \item Built a numerical physics engine simulating gravitational interactions of three bodies in 2D.
    \item Implemented ODE solvers and visualization of orbital trajectories.
\end{itemize}

%-------------------- LEADERSHIP & ACTIVITIES --------------------
\section*{Leadership \& Activities}
\textbf{Formula Slug} --- Software + Firmware Engineering Member \\
\textbf{Association for Computing Machinery (ACM)} --- Member \\
\textbf{Google Developer Groups on Campus} --- Former Instruction Officer, Workshop Organizer

%-------------------- SKILLS --------------------
\section*{Skills}
\begin{tabular}{@{}ll}
\textbf{Languages:} & C/C++, Python, Java, MATLAB, Node.js, Bash \\
\textbf{Embedded/HW:} & RTOS (MbedOS), STM32, CAN bus, UART, SPI, KiCad, Oscilloscopes \\
\textbf{Tools/DevOps:} & Git, CMake, Ninja, Docker, Valgrind, GNU Make, GDB, Vim \\
\textbf{Math/Theory:} & DSP, FFT, Numerical Integration, ODE Solvers, PID Control \\
\textbf{AI/Productivity:} & LLM Prompt Engineering (Copilot, Gemini, Grok), Technical Documentation
\end{tabular}

\end{document}
